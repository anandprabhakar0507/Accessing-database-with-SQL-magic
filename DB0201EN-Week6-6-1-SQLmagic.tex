
% Default to the notebook output style

    


% Inherit from the specified cell style.




    
\documentclass[11pt]{article}

    
    
    \usepackage[T1]{fontenc}
    % Nicer default font (+ math font) than Computer Modern for most use cases
    \usepackage{mathpazo}

    % Basic figure setup, for now with no caption control since it's done
    % automatically by Pandoc (which extracts ![](path) syntax from Markdown).
    \usepackage{graphicx}
    % We will generate all images so they have a width \maxwidth. This means
    % that they will get their normal width if they fit onto the page, but
    % are scaled down if they would overflow the margins.
    \makeatletter
    \def\maxwidth{\ifdim\Gin@nat@width>\linewidth\linewidth
    \else\Gin@nat@width\fi}
    \makeatother
    \let\Oldincludegraphics\includegraphics
    % Set max figure width to be 80% of text width, for now hardcoded.
    \renewcommand{\includegraphics}[1]{\Oldincludegraphics[width=.8\maxwidth]{#1}}
    % Ensure that by default, figures have no caption (until we provide a
    % proper Figure object with a Caption API and a way to capture that
    % in the conversion process - todo).
    \usepackage{caption}
    \DeclareCaptionLabelFormat{nolabel}{}
    \captionsetup{labelformat=nolabel}

    \usepackage{adjustbox} % Used to constrain images to a maximum size 
    \usepackage{xcolor} % Allow colors to be defined
    \usepackage{enumerate} % Needed for markdown enumerations to work
    \usepackage{geometry} % Used to adjust the document margins
    \usepackage{amsmath} % Equations
    \usepackage{amssymb} % Equations
    \usepackage{textcomp} % defines textquotesingle
    % Hack from http://tex.stackexchange.com/a/47451/13684:
    \AtBeginDocument{%
        \def\PYZsq{\textquotesingle}% Upright quotes in Pygmentized code
    }
    \usepackage{upquote} % Upright quotes for verbatim code
    \usepackage{eurosym} % defines \euro
    \usepackage[mathletters]{ucs} % Extended unicode (utf-8) support
    \usepackage[utf8x]{inputenc} % Allow utf-8 characters in the tex document
    \usepackage{fancyvrb} % verbatim replacement that allows latex
    \usepackage{grffile} % extends the file name processing of package graphics 
                         % to support a larger range 
    % The hyperref package gives us a pdf with properly built
    % internal navigation ('pdf bookmarks' for the table of contents,
    % internal cross-reference links, web links for URLs, etc.)
    \usepackage{hyperref}
    \usepackage{longtable} % longtable support required by pandoc >1.10
    \usepackage{booktabs}  % table support for pandoc > 1.12.2
    \usepackage[inline]{enumitem} % IRkernel/repr support (it uses the enumerate* environment)
    \usepackage[normalem]{ulem} % ulem is needed to support strikethroughs (\sout)
                                % normalem makes italics be italics, not underlines
    

    
    
    % Colors for the hyperref package
    \definecolor{urlcolor}{rgb}{0,.145,.698}
    \definecolor{linkcolor}{rgb}{.71,0.21,0.01}
    \definecolor{citecolor}{rgb}{.12,.54,.11}

    % ANSI colors
    \definecolor{ansi-black}{HTML}{3E424D}
    \definecolor{ansi-black-intense}{HTML}{282C36}
    \definecolor{ansi-red}{HTML}{E75C58}
    \definecolor{ansi-red-intense}{HTML}{B22B31}
    \definecolor{ansi-green}{HTML}{00A250}
    \definecolor{ansi-green-intense}{HTML}{007427}
    \definecolor{ansi-yellow}{HTML}{DDB62B}
    \definecolor{ansi-yellow-intense}{HTML}{B27D12}
    \definecolor{ansi-blue}{HTML}{208FFB}
    \definecolor{ansi-blue-intense}{HTML}{0065CA}
    \definecolor{ansi-magenta}{HTML}{D160C4}
    \definecolor{ansi-magenta-intense}{HTML}{A03196}
    \definecolor{ansi-cyan}{HTML}{60C6C8}
    \definecolor{ansi-cyan-intense}{HTML}{258F8F}
    \definecolor{ansi-white}{HTML}{C5C1B4}
    \definecolor{ansi-white-intense}{HTML}{A1A6B2}

    % commands and environments needed by pandoc snippets
    % extracted from the output of `pandoc -s`
    \providecommand{\tightlist}{%
      \setlength{\itemsep}{0pt}\setlength{\parskip}{0pt}}
    \DefineVerbatimEnvironment{Highlighting}{Verbatim}{commandchars=\\\{\}}
    % Add ',fontsize=\small' for more characters per line
    \newenvironment{Shaded}{}{}
    \newcommand{\KeywordTok}[1]{\textcolor[rgb]{0.00,0.44,0.13}{\textbf{{#1}}}}
    \newcommand{\DataTypeTok}[1]{\textcolor[rgb]{0.56,0.13,0.00}{{#1}}}
    \newcommand{\DecValTok}[1]{\textcolor[rgb]{0.25,0.63,0.44}{{#1}}}
    \newcommand{\BaseNTok}[1]{\textcolor[rgb]{0.25,0.63,0.44}{{#1}}}
    \newcommand{\FloatTok}[1]{\textcolor[rgb]{0.25,0.63,0.44}{{#1}}}
    \newcommand{\CharTok}[1]{\textcolor[rgb]{0.25,0.44,0.63}{{#1}}}
    \newcommand{\StringTok}[1]{\textcolor[rgb]{0.25,0.44,0.63}{{#1}}}
    \newcommand{\CommentTok}[1]{\textcolor[rgb]{0.38,0.63,0.69}{\textit{{#1}}}}
    \newcommand{\OtherTok}[1]{\textcolor[rgb]{0.00,0.44,0.13}{{#1}}}
    \newcommand{\AlertTok}[1]{\textcolor[rgb]{1.00,0.00,0.00}{\textbf{{#1}}}}
    \newcommand{\FunctionTok}[1]{\textcolor[rgb]{0.02,0.16,0.49}{{#1}}}
    \newcommand{\RegionMarkerTok}[1]{{#1}}
    \newcommand{\ErrorTok}[1]{\textcolor[rgb]{1.00,0.00,0.00}{\textbf{{#1}}}}
    \newcommand{\NormalTok}[1]{{#1}}
    
    % Additional commands for more recent versions of Pandoc
    \newcommand{\ConstantTok}[1]{\textcolor[rgb]{0.53,0.00,0.00}{{#1}}}
    \newcommand{\SpecialCharTok}[1]{\textcolor[rgb]{0.25,0.44,0.63}{{#1}}}
    \newcommand{\VerbatimStringTok}[1]{\textcolor[rgb]{0.25,0.44,0.63}{{#1}}}
    \newcommand{\SpecialStringTok}[1]{\textcolor[rgb]{0.73,0.40,0.53}{{#1}}}
    \newcommand{\ImportTok}[1]{{#1}}
    \newcommand{\DocumentationTok}[1]{\textcolor[rgb]{0.73,0.13,0.13}{\textit{{#1}}}}
    \newcommand{\AnnotationTok}[1]{\textcolor[rgb]{0.38,0.63,0.69}{\textbf{\textit{{#1}}}}}
    \newcommand{\CommentVarTok}[1]{\textcolor[rgb]{0.38,0.63,0.69}{\textbf{\textit{{#1}}}}}
    \newcommand{\VariableTok}[1]{\textcolor[rgb]{0.10,0.09,0.49}{{#1}}}
    \newcommand{\ControlFlowTok}[1]{\textcolor[rgb]{0.00,0.44,0.13}{\textbf{{#1}}}}
    \newcommand{\OperatorTok}[1]{\textcolor[rgb]{0.40,0.40,0.40}{{#1}}}
    \newcommand{\BuiltInTok}[1]{{#1}}
    \newcommand{\ExtensionTok}[1]{{#1}}
    \newcommand{\PreprocessorTok}[1]{\textcolor[rgb]{0.74,0.48,0.00}{{#1}}}
    \newcommand{\AttributeTok}[1]{\textcolor[rgb]{0.49,0.56,0.16}{{#1}}}
    \newcommand{\InformationTok}[1]{\textcolor[rgb]{0.38,0.63,0.69}{\textbf{\textit{{#1}}}}}
    \newcommand{\WarningTok}[1]{\textcolor[rgb]{0.38,0.63,0.69}{\textbf{\textit{{#1}}}}}
    
    
    % Define a nice break command that doesn't care if a line doesn't already
    % exist.
    \def\br{\hspace*{\fill} \\* }
    % Math Jax compatability definitions
    \def\gt{>}
    \def\lt{<}
    % Document parameters
    \title{DB0201EN-Week6-6-1-SQLmagic}
    
    
    

    % Pygments definitions
    
\makeatletter
\def\PY@reset{\let\PY@it=\relax \let\PY@bf=\relax%
    \let\PY@ul=\relax \let\PY@tc=\relax%
    \let\PY@bc=\relax \let\PY@ff=\relax}
\def\PY@tok#1{\csname PY@tok@#1\endcsname}
\def\PY@toks#1+{\ifx\relax#1\empty\else%
    \PY@tok{#1}\expandafter\PY@toks\fi}
\def\PY@do#1{\PY@bc{\PY@tc{\PY@ul{%
    \PY@it{\PY@bf{\PY@ff{#1}}}}}}}
\def\PY#1#2{\PY@reset\PY@toks#1+\relax+\PY@do{#2}}

\expandafter\def\csname PY@tok@w\endcsname{\def\PY@tc##1{\textcolor[rgb]{0.73,0.73,0.73}{##1}}}
\expandafter\def\csname PY@tok@c\endcsname{\let\PY@it=\textit\def\PY@tc##1{\textcolor[rgb]{0.25,0.50,0.50}{##1}}}
\expandafter\def\csname PY@tok@cp\endcsname{\def\PY@tc##1{\textcolor[rgb]{0.74,0.48,0.00}{##1}}}
\expandafter\def\csname PY@tok@k\endcsname{\let\PY@bf=\textbf\def\PY@tc##1{\textcolor[rgb]{0.00,0.50,0.00}{##1}}}
\expandafter\def\csname PY@tok@kp\endcsname{\def\PY@tc##1{\textcolor[rgb]{0.00,0.50,0.00}{##1}}}
\expandafter\def\csname PY@tok@kt\endcsname{\def\PY@tc##1{\textcolor[rgb]{0.69,0.00,0.25}{##1}}}
\expandafter\def\csname PY@tok@o\endcsname{\def\PY@tc##1{\textcolor[rgb]{0.40,0.40,0.40}{##1}}}
\expandafter\def\csname PY@tok@ow\endcsname{\let\PY@bf=\textbf\def\PY@tc##1{\textcolor[rgb]{0.67,0.13,1.00}{##1}}}
\expandafter\def\csname PY@tok@nb\endcsname{\def\PY@tc##1{\textcolor[rgb]{0.00,0.50,0.00}{##1}}}
\expandafter\def\csname PY@tok@nf\endcsname{\def\PY@tc##1{\textcolor[rgb]{0.00,0.00,1.00}{##1}}}
\expandafter\def\csname PY@tok@nc\endcsname{\let\PY@bf=\textbf\def\PY@tc##1{\textcolor[rgb]{0.00,0.00,1.00}{##1}}}
\expandafter\def\csname PY@tok@nn\endcsname{\let\PY@bf=\textbf\def\PY@tc##1{\textcolor[rgb]{0.00,0.00,1.00}{##1}}}
\expandafter\def\csname PY@tok@ne\endcsname{\let\PY@bf=\textbf\def\PY@tc##1{\textcolor[rgb]{0.82,0.25,0.23}{##1}}}
\expandafter\def\csname PY@tok@nv\endcsname{\def\PY@tc##1{\textcolor[rgb]{0.10,0.09,0.49}{##1}}}
\expandafter\def\csname PY@tok@no\endcsname{\def\PY@tc##1{\textcolor[rgb]{0.53,0.00,0.00}{##1}}}
\expandafter\def\csname PY@tok@nl\endcsname{\def\PY@tc##1{\textcolor[rgb]{0.63,0.63,0.00}{##1}}}
\expandafter\def\csname PY@tok@ni\endcsname{\let\PY@bf=\textbf\def\PY@tc##1{\textcolor[rgb]{0.60,0.60,0.60}{##1}}}
\expandafter\def\csname PY@tok@na\endcsname{\def\PY@tc##1{\textcolor[rgb]{0.49,0.56,0.16}{##1}}}
\expandafter\def\csname PY@tok@nt\endcsname{\let\PY@bf=\textbf\def\PY@tc##1{\textcolor[rgb]{0.00,0.50,0.00}{##1}}}
\expandafter\def\csname PY@tok@nd\endcsname{\def\PY@tc##1{\textcolor[rgb]{0.67,0.13,1.00}{##1}}}
\expandafter\def\csname PY@tok@s\endcsname{\def\PY@tc##1{\textcolor[rgb]{0.73,0.13,0.13}{##1}}}
\expandafter\def\csname PY@tok@sd\endcsname{\let\PY@it=\textit\def\PY@tc##1{\textcolor[rgb]{0.73,0.13,0.13}{##1}}}
\expandafter\def\csname PY@tok@si\endcsname{\let\PY@bf=\textbf\def\PY@tc##1{\textcolor[rgb]{0.73,0.40,0.53}{##1}}}
\expandafter\def\csname PY@tok@se\endcsname{\let\PY@bf=\textbf\def\PY@tc##1{\textcolor[rgb]{0.73,0.40,0.13}{##1}}}
\expandafter\def\csname PY@tok@sr\endcsname{\def\PY@tc##1{\textcolor[rgb]{0.73,0.40,0.53}{##1}}}
\expandafter\def\csname PY@tok@ss\endcsname{\def\PY@tc##1{\textcolor[rgb]{0.10,0.09,0.49}{##1}}}
\expandafter\def\csname PY@tok@sx\endcsname{\def\PY@tc##1{\textcolor[rgb]{0.00,0.50,0.00}{##1}}}
\expandafter\def\csname PY@tok@m\endcsname{\def\PY@tc##1{\textcolor[rgb]{0.40,0.40,0.40}{##1}}}
\expandafter\def\csname PY@tok@gh\endcsname{\let\PY@bf=\textbf\def\PY@tc##1{\textcolor[rgb]{0.00,0.00,0.50}{##1}}}
\expandafter\def\csname PY@tok@gu\endcsname{\let\PY@bf=\textbf\def\PY@tc##1{\textcolor[rgb]{0.50,0.00,0.50}{##1}}}
\expandafter\def\csname PY@tok@gd\endcsname{\def\PY@tc##1{\textcolor[rgb]{0.63,0.00,0.00}{##1}}}
\expandafter\def\csname PY@tok@gi\endcsname{\def\PY@tc##1{\textcolor[rgb]{0.00,0.63,0.00}{##1}}}
\expandafter\def\csname PY@tok@gr\endcsname{\def\PY@tc##1{\textcolor[rgb]{1.00,0.00,0.00}{##1}}}
\expandafter\def\csname PY@tok@ge\endcsname{\let\PY@it=\textit}
\expandafter\def\csname PY@tok@gs\endcsname{\let\PY@bf=\textbf}
\expandafter\def\csname PY@tok@gp\endcsname{\let\PY@bf=\textbf\def\PY@tc##1{\textcolor[rgb]{0.00,0.00,0.50}{##1}}}
\expandafter\def\csname PY@tok@go\endcsname{\def\PY@tc##1{\textcolor[rgb]{0.53,0.53,0.53}{##1}}}
\expandafter\def\csname PY@tok@gt\endcsname{\def\PY@tc##1{\textcolor[rgb]{0.00,0.27,0.87}{##1}}}
\expandafter\def\csname PY@tok@err\endcsname{\def\PY@bc##1{\setlength{\fboxsep}{0pt}\fcolorbox[rgb]{1.00,0.00,0.00}{1,1,1}{\strut ##1}}}
\expandafter\def\csname PY@tok@kc\endcsname{\let\PY@bf=\textbf\def\PY@tc##1{\textcolor[rgb]{0.00,0.50,0.00}{##1}}}
\expandafter\def\csname PY@tok@kd\endcsname{\let\PY@bf=\textbf\def\PY@tc##1{\textcolor[rgb]{0.00,0.50,0.00}{##1}}}
\expandafter\def\csname PY@tok@kn\endcsname{\let\PY@bf=\textbf\def\PY@tc##1{\textcolor[rgb]{0.00,0.50,0.00}{##1}}}
\expandafter\def\csname PY@tok@kr\endcsname{\let\PY@bf=\textbf\def\PY@tc##1{\textcolor[rgb]{0.00,0.50,0.00}{##1}}}
\expandafter\def\csname PY@tok@bp\endcsname{\def\PY@tc##1{\textcolor[rgb]{0.00,0.50,0.00}{##1}}}
\expandafter\def\csname PY@tok@fm\endcsname{\def\PY@tc##1{\textcolor[rgb]{0.00,0.00,1.00}{##1}}}
\expandafter\def\csname PY@tok@vc\endcsname{\def\PY@tc##1{\textcolor[rgb]{0.10,0.09,0.49}{##1}}}
\expandafter\def\csname PY@tok@vg\endcsname{\def\PY@tc##1{\textcolor[rgb]{0.10,0.09,0.49}{##1}}}
\expandafter\def\csname PY@tok@vi\endcsname{\def\PY@tc##1{\textcolor[rgb]{0.10,0.09,0.49}{##1}}}
\expandafter\def\csname PY@tok@vm\endcsname{\def\PY@tc##1{\textcolor[rgb]{0.10,0.09,0.49}{##1}}}
\expandafter\def\csname PY@tok@sa\endcsname{\def\PY@tc##1{\textcolor[rgb]{0.73,0.13,0.13}{##1}}}
\expandafter\def\csname PY@tok@sb\endcsname{\def\PY@tc##1{\textcolor[rgb]{0.73,0.13,0.13}{##1}}}
\expandafter\def\csname PY@tok@sc\endcsname{\def\PY@tc##1{\textcolor[rgb]{0.73,0.13,0.13}{##1}}}
\expandafter\def\csname PY@tok@dl\endcsname{\def\PY@tc##1{\textcolor[rgb]{0.73,0.13,0.13}{##1}}}
\expandafter\def\csname PY@tok@s2\endcsname{\def\PY@tc##1{\textcolor[rgb]{0.73,0.13,0.13}{##1}}}
\expandafter\def\csname PY@tok@sh\endcsname{\def\PY@tc##1{\textcolor[rgb]{0.73,0.13,0.13}{##1}}}
\expandafter\def\csname PY@tok@s1\endcsname{\def\PY@tc##1{\textcolor[rgb]{0.73,0.13,0.13}{##1}}}
\expandafter\def\csname PY@tok@mb\endcsname{\def\PY@tc##1{\textcolor[rgb]{0.40,0.40,0.40}{##1}}}
\expandafter\def\csname PY@tok@mf\endcsname{\def\PY@tc##1{\textcolor[rgb]{0.40,0.40,0.40}{##1}}}
\expandafter\def\csname PY@tok@mh\endcsname{\def\PY@tc##1{\textcolor[rgb]{0.40,0.40,0.40}{##1}}}
\expandafter\def\csname PY@tok@mi\endcsname{\def\PY@tc##1{\textcolor[rgb]{0.40,0.40,0.40}{##1}}}
\expandafter\def\csname PY@tok@il\endcsname{\def\PY@tc##1{\textcolor[rgb]{0.40,0.40,0.40}{##1}}}
\expandafter\def\csname PY@tok@mo\endcsname{\def\PY@tc##1{\textcolor[rgb]{0.40,0.40,0.40}{##1}}}
\expandafter\def\csname PY@tok@ch\endcsname{\let\PY@it=\textit\def\PY@tc##1{\textcolor[rgb]{0.25,0.50,0.50}{##1}}}
\expandafter\def\csname PY@tok@cm\endcsname{\let\PY@it=\textit\def\PY@tc##1{\textcolor[rgb]{0.25,0.50,0.50}{##1}}}
\expandafter\def\csname PY@tok@cpf\endcsname{\let\PY@it=\textit\def\PY@tc##1{\textcolor[rgb]{0.25,0.50,0.50}{##1}}}
\expandafter\def\csname PY@tok@c1\endcsname{\let\PY@it=\textit\def\PY@tc##1{\textcolor[rgb]{0.25,0.50,0.50}{##1}}}
\expandafter\def\csname PY@tok@cs\endcsname{\let\PY@it=\textit\def\PY@tc##1{\textcolor[rgb]{0.25,0.50,0.50}{##1}}}

\def\PYZbs{\char`\\}
\def\PYZus{\char`\_}
\def\PYZob{\char`\{}
\def\PYZcb{\char`\}}
\def\PYZca{\char`\^}
\def\PYZam{\char`\&}
\def\PYZlt{\char`\<}
\def\PYZgt{\char`\>}
\def\PYZsh{\char`\#}
\def\PYZpc{\char`\%}
\def\PYZdl{\char`\$}
\def\PYZhy{\char`\-}
\def\PYZsq{\char`\'}
\def\PYZdq{\char`\"}
\def\PYZti{\char`\~}
% for compatibility with earlier versions
\def\PYZat{@}
\def\PYZlb{[}
\def\PYZrb{]}
\makeatother


    % Exact colors from NB
    \definecolor{incolor}{rgb}{0.0, 0.0, 0.5}
    \definecolor{outcolor}{rgb}{0.545, 0.0, 0.0}



    
    % Prevent overflowing lines due to hard-to-break entities
    \sloppy 
    % Setup hyperref package
    \hypersetup{
      breaklinks=true,  % so long urls are correctly broken across lines
      colorlinks=true,
      urlcolor=urlcolor,
      linkcolor=linkcolor,
      citecolor=citecolor,
      }
    % Slightly bigger margins than the latex defaults
    
    \geometry{verbose,tmargin=1in,bmargin=1in,lmargin=1in,rmargin=1in}
    
    

    \begin{document}
    
    
    \maketitle
    
    

    
    Accessing Databases with SQL Magic

    \paragraph{After using this notebook, you will know how to perform
simplified database access using SQL "magic". You will connect to a Db2
database, issue SQL commands to create tables, insert data, and run
queries, as well as retrieve results in a Python
dataframe.}\label{after-using-this-notebook-you-will-know-how-to-perform-simplified-database-access-using-sql-magic.-you-will-connect-to-a-db2-database-issue-sql-commands-to-create-tables-insert-data-and-run-queries-as-well-as-retrieve-results-in-a-python-dataframe.}

    \subparagraph{\texorpdfstring{To communicate with SQL Databases from
within a JupyterLab notebook, we can use the SQL "magic" provided by the
\href{https://github.com/catherinedevlin/ipython-sql}{ipython-sql}
extension. "Magic" is JupyterLab's term for special commands that start
with "\%". Below, we'll use the \emph{load}\_\emph{ext} magic to load
the ipython-sql
extension.}{To communicate with SQL Databases from within a JupyterLab notebook, we can use the SQL "magic" provided by the ipython-sql extension. "Magic" is JupyterLab's term for special commands that start with "\%". Below, we'll use the load\_ext magic to load the ipython-sql extension.}}\label{to-communicate-with-sql-databases-from-within-a-jupyterlab-notebook-we-can-use-the-sql-magic-provided-by-the-ipython-sql-extension.-magic-is-jupyterlabs-term-for-special-commands-that-start-with-.-below-well-use-the-load_ext-magic-to-load-the-ipython-sql-extension.}

    \begin{Verbatim}[commandchars=\\\{\}]
{\color{incolor}In [{\color{incolor} }]:} \PY{o}{\PYZpc{}}\PY{k}{load\PYZus{}ext} sql
\end{Verbatim}


    \subparagraph{Now we have access to SQL magic. With our first SQL magic
command, we'll connect to a Db2 database. However, in order to do that,
you'll first need to retrieve or create your credentials to access your
Db2
database.}\label{now-we-have-access-to-sql-magic.-with-our-first-sql-magic-command-well-connect-to-a-db2-database.-however-in-order-to-do-that-youll-first-need-to-retrieve-or-create-your-credentials-to-access-your-db2-database.}

    This image shows the location of your connection string if you're using
Db2 on IBM Cloud. If you're using another host the format is:
username:password@hostname:port/database-name

    \begin{Verbatim}[commandchars=\\\{\}]
{\color{incolor}In [{\color{incolor} }]:} \PY{c+c1}{\PYZsh{} Note the ibm\PYZus{}db\PYZus{}sa:// prefix instead of db://}
        \PY{c+c1}{\PYZsh{} This is because JupyterLab\PYZsq{}s ipython\PYZhy{}sql extension uses sqlalchemy (a python SQL toolkit)}
        \PY{c+c1}{\PYZsh{} which in turn uses IBM\PYZsq{}s sqlalchemy dialect: ibm\PYZus{}db\PYZus{}sa}
        \PY{c+c1}{\PYZsh{} substitute your Db2 credentials in the connection string below}
        
        \PY{o}{\PYZpc{}}\PY{k}{sql} ibm\PYZus{}db\PYZus{}sa://my\PYZhy{}username:my\PYZhy{}password@my\PYZhy{}hostname:my\PYZhy{}port/my\PYZhy{}db\PYZhy{}name
\end{Verbatim}


    \subparagraph{For convenience, we can use \%\%sql (two \%'s instead of
one) at the top of a cell to indicate we want the entire cell to be
treated as SQL. Let's use this to create a table and fill it with some
test data for
experimenting.}\label{for-convenience-we-can-use-sql-two-s-instead-of-one-at-the-top-of-a-cell-to-indicate-we-want-the-entire-cell-to-be-treated-as-sql.-lets-use-this-to-create-a-table-and-fill-it-with-some-test-data-for-experimenting.}

    \begin{Verbatim}[commandchars=\\\{\}]
{\color{incolor}In [{\color{incolor} }]:} \PY{o}{\PYZpc{}\PYZpc{}}\PY{k}{sql}
        
        CREATE TABLE INTERNATIONAL\PYZus{}STUDENT\PYZus{}TEST\PYZus{}SCORES (
        	country VARCHAR(50),
        	first\PYZus{}name VARCHAR(50),
        	last\PYZus{}name VARCHAR(50),
        	test\PYZus{}score INT
        );
        INSERT INTO INTERNATIONAL\PYZus{}STUDENT\PYZus{}TEST\PYZus{}SCORES (country, first\PYZus{}name, last\PYZus{}name, test\PYZus{}score)
        VALUES
        (\PYZsq{}United States\PYZsq{}, \PYZsq{}Marshall\PYZsq{}, \PYZsq{}Bernadot\PYZsq{}, 54),
        (\PYZsq{}Ghana\PYZsq{}, \PYZsq{}Celinda\PYZsq{}, \PYZsq{}Malkin\PYZsq{}, 51),
        (\PYZsq{}Ukraine\PYZsq{}, \PYZsq{}Guillermo\PYZsq{}, \PYZsq{}Furze\PYZsq{}, 53),
        (\PYZsq{}Greece\PYZsq{}, \PYZsq{}Aharon\PYZsq{}, \PYZsq{}Tunnow\PYZsq{}, 48),
        (\PYZsq{}Russia\PYZsq{}, \PYZsq{}Bail\PYZsq{}, \PYZsq{}Goodwin\PYZsq{}, 46),
        (\PYZsq{}Poland\PYZsq{}, \PYZsq{}Cole\PYZsq{}, \PYZsq{}Winteringham\PYZsq{}, 49),
        (\PYZsq{}Sweden\PYZsq{}, \PYZsq{}Emlyn\PYZsq{}, \PYZsq{}Erricker\PYZsq{}, 55),
        (\PYZsq{}Russia\PYZsq{}, \PYZsq{}Cathee\PYZsq{}, \PYZsq{}Sivewright\PYZsq{}, 49),
        (\PYZsq{}China\PYZsq{}, \PYZsq{}Barny\PYZsq{}, \PYZsq{}Ingerson\PYZsq{}, 57),
        (\PYZsq{}Uganda\PYZsq{}, \PYZsq{}Sharla\PYZsq{}, \PYZsq{}Papaccio\PYZsq{}, 55),
        (\PYZsq{}China\PYZsq{}, \PYZsq{}Stella\PYZsq{}, \PYZsq{}Youens\PYZsq{}, 51),
        (\PYZsq{}Poland\PYZsq{}, \PYZsq{}Julio\PYZsq{}, \PYZsq{}Buesden\PYZsq{}, 48),
        (\PYZsq{}United States\PYZsq{}, \PYZsq{}Tiffie\PYZsq{}, \PYZsq{}Cosely\PYZsq{}, 58),
        (\PYZsq{}Poland\PYZsq{}, \PYZsq{}Auroora\PYZsq{}, \PYZsq{}Stiffell\PYZsq{}, 45),
        (\PYZsq{}China\PYZsq{}, \PYZsq{}Clarita\PYZsq{}, \PYZsq{}Huet\PYZsq{}, 52),
        (\PYZsq{}Poland\PYZsq{}, \PYZsq{}Shannon\PYZsq{}, \PYZsq{}Goulden\PYZsq{}, 45),
        (\PYZsq{}Philippines\PYZsq{}, \PYZsq{}Emylee\PYZsq{}, \PYZsq{}Privost\PYZsq{}, 50),
        (\PYZsq{}France\PYZsq{}, \PYZsq{}Madelina\PYZsq{}, \PYZsq{}Burk\PYZsq{}, 49),
        (\PYZsq{}China\PYZsq{}, \PYZsq{}Saunderson\PYZsq{}, \PYZsq{}Root\PYZsq{}, 58),
        (\PYZsq{}Indonesia\PYZsq{}, \PYZsq{}Bo\PYZsq{}, \PYZsq{}Waring\PYZsq{}, 55),
        (\PYZsq{}China\PYZsq{}, \PYZsq{}Hollis\PYZsq{}, \PYZsq{}Domotor\PYZsq{}, 45),
        (\PYZsq{}Russia\PYZsq{}, \PYZsq{}Robbie\PYZsq{}, \PYZsq{}Collip\PYZsq{}, 46),
        (\PYZsq{}Philippines\PYZsq{}, \PYZsq{}Davon\PYZsq{}, \PYZsq{}Donisi\PYZsq{}, 46),
        (\PYZsq{}China\PYZsq{}, \PYZsq{}Cristabel\PYZsq{}, \PYZsq{}Radeliffe\PYZsq{}, 48),
        (\PYZsq{}China\PYZsq{}, \PYZsq{}Wallis\PYZsq{}, \PYZsq{}Bartleet\PYZsq{}, 58),
        (\PYZsq{}Moldova\PYZsq{}, \PYZsq{}Arleen\PYZsq{}, \PYZsq{}Stailey\PYZsq{}, 38),
        (\PYZsq{}Ireland\PYZsq{}, \PYZsq{}Mendel\PYZsq{}, \PYZsq{}Grumble\PYZsq{}, 58),
        (\PYZsq{}China\PYZsq{}, \PYZsq{}Sallyann\PYZsq{}, \PYZsq{}Exley\PYZsq{}, 51),
        (\PYZsq{}Mexico\PYZsq{}, \PYZsq{}Kain\PYZsq{}, \PYZsq{}Swaite\PYZsq{}, 46),
        (\PYZsq{}Indonesia\PYZsq{}, \PYZsq{}Alonso\PYZsq{}, \PYZsq{}Bulteel\PYZsq{}, 45),
        (\PYZsq{}Armenia\PYZsq{}, \PYZsq{}Anatol\PYZsq{}, \PYZsq{}Tankus\PYZsq{}, 51),
        (\PYZsq{}Indonesia\PYZsq{}, \PYZsq{}Coralyn\PYZsq{}, \PYZsq{}Dawkins\PYZsq{}, 48),
        (\PYZsq{}China\PYZsq{}, \PYZsq{}Deanne\PYZsq{}, \PYZsq{}Edwinson\PYZsq{}, 45),
        (\PYZsq{}China\PYZsq{}, \PYZsq{}Georgiana\PYZsq{}, \PYZsq{}Epple\PYZsq{}, 51),
        (\PYZsq{}Portugal\PYZsq{}, \PYZsq{}Bartlet\PYZsq{}, \PYZsq{}Breese\PYZsq{}, 56),
        (\PYZsq{}Azerbaijan\PYZsq{}, \PYZsq{}Idalina\PYZsq{}, \PYZsq{}Lukash\PYZsq{}, 50),
        (\PYZsq{}France\PYZsq{}, \PYZsq{}Livvie\PYZsq{}, \PYZsq{}Flory\PYZsq{}, 54),
        (\PYZsq{}Malaysia\PYZsq{}, \PYZsq{}Nonie\PYZsq{}, \PYZsq{}Borit\PYZsq{}, 48),
        (\PYZsq{}Indonesia\PYZsq{}, \PYZsq{}Clio\PYZsq{}, \PYZsq{}Mugg\PYZsq{}, 47),
        (\PYZsq{}Brazil\PYZsq{}, \PYZsq{}Westley\PYZsq{}, \PYZsq{}Measor\PYZsq{}, 48),
        (\PYZsq{}Philippines\PYZsq{}, \PYZsq{}Katrinka\PYZsq{}, \PYZsq{}Sibbert\PYZsq{}, 51),
        (\PYZsq{}Poland\PYZsq{}, \PYZsq{}Valentia\PYZsq{}, \PYZsq{}Mounch\PYZsq{}, 50),
        (\PYZsq{}Norway\PYZsq{}, \PYZsq{}Sheilah\PYZsq{}, \PYZsq{}Hedditch\PYZsq{}, 53),
        (\PYZsq{}Papua New Guinea\PYZsq{}, \PYZsq{}Itch\PYZsq{}, \PYZsq{}Jubb\PYZsq{}, 50),
        (\PYZsq{}Latvia\PYZsq{}, \PYZsq{}Stesha\PYZsq{}, \PYZsq{}Garnson\PYZsq{}, 53),
        (\PYZsq{}Canada\PYZsq{}, \PYZsq{}Cristionna\PYZsq{}, \PYZsq{}Wadmore\PYZsq{}, 46),
        (\PYZsq{}China\PYZsq{}, \PYZsq{}Lianna\PYZsq{}, \PYZsq{}Gatward\PYZsq{}, 43),
        (\PYZsq{}Guatemala\PYZsq{}, \PYZsq{}Tanney\PYZsq{}, \PYZsq{}Vials\PYZsq{}, 48),
        (\PYZsq{}France\PYZsq{}, \PYZsq{}Alma\PYZsq{}, \PYZsq{}Zavittieri\PYZsq{}, 44),
        (\PYZsq{}China\PYZsq{}, \PYZsq{}Alvira\PYZsq{}, \PYZsq{}Tamas\PYZsq{}, 50),
        (\PYZsq{}United States\PYZsq{}, \PYZsq{}Shanon\PYZsq{}, \PYZsq{}Peres\PYZsq{}, 45),
        (\PYZsq{}Sweden\PYZsq{}, \PYZsq{}Maisey\PYZsq{}, \PYZsq{}Lynas\PYZsq{}, 53),
        (\PYZsq{}Indonesia\PYZsq{}, \PYZsq{}Kip\PYZsq{}, \PYZsq{}Hothersall\PYZsq{}, 46),
        (\PYZsq{}China\PYZsq{}, \PYZsq{}Cash\PYZsq{}, \PYZsq{}Landis\PYZsq{}, 48),
        (\PYZsq{}Panama\PYZsq{}, \PYZsq{}Kennith\PYZsq{}, \PYZsq{}Digance\PYZsq{}, 45),
        (\PYZsq{}China\PYZsq{}, \PYZsq{}Ulberto\PYZsq{}, \PYZsq{}Riggeard\PYZsq{}, 48),
        (\PYZsq{}Switzerland\PYZsq{}, \PYZsq{}Judy\PYZsq{}, \PYZsq{}Gilligan\PYZsq{}, 49),
        (\PYZsq{}Philippines\PYZsq{}, \PYZsq{}Tod\PYZsq{}, \PYZsq{}Trevaskus\PYZsq{}, 52),
        (\PYZsq{}Brazil\PYZsq{}, \PYZsq{}Herold\PYZsq{}, \PYZsq{}Heggs\PYZsq{}, 44),
        (\PYZsq{}Latvia\PYZsq{}, \PYZsq{}Verney\PYZsq{}, \PYZsq{}Note\PYZsq{}, 50),
        (\PYZsq{}Poland\PYZsq{}, \PYZsq{}Temp\PYZsq{}, \PYZsq{}Ribey\PYZsq{}, 50),
        (\PYZsq{}China\PYZsq{}, \PYZsq{}Conroy\PYZsq{}, \PYZsq{}Egdal\PYZsq{}, 48),
        (\PYZsq{}Japan\PYZsq{}, \PYZsq{}Gabie\PYZsq{}, \PYZsq{}Alessandone\PYZsq{}, 47),
        (\PYZsq{}Ukraine\PYZsq{}, \PYZsq{}Devlen\PYZsq{}, \PYZsq{}Chaperlin\PYZsq{}, 54),
        (\PYZsq{}France\PYZsq{}, \PYZsq{}Babbette\PYZsq{}, \PYZsq{}Turner\PYZsq{}, 51),
        (\PYZsq{}Czech Republic\PYZsq{}, \PYZsq{}Virgil\PYZsq{}, \PYZsq{}Scotney\PYZsq{}, 52),
        (\PYZsq{}Tajikistan\PYZsq{}, \PYZsq{}Zorina\PYZsq{}, \PYZsq{}Bedow\PYZsq{}, 49),
        (\PYZsq{}China\PYZsq{}, \PYZsq{}Aidan\PYZsq{}, \PYZsq{}Rudeyeard\PYZsq{}, 50),
        (\PYZsq{}Ireland\PYZsq{}, \PYZsq{}Saunder\PYZsq{}, \PYZsq{}MacLice\PYZsq{}, 48),
        (\PYZsq{}France\PYZsq{}, \PYZsq{}Waly\PYZsq{}, \PYZsq{}Brunstan\PYZsq{}, 53),
        (\PYZsq{}China\PYZsq{}, \PYZsq{}Gisele\PYZsq{}, \PYZsq{}Enns\PYZsq{}, 52),
        (\PYZsq{}Peru\PYZsq{}, \PYZsq{}Mina\PYZsq{}, \PYZsq{}Winchester\PYZsq{}, 48),
        (\PYZsq{}Japan\PYZsq{}, \PYZsq{}Torie\PYZsq{}, \PYZsq{}MacShirrie\PYZsq{}, 50),
        (\PYZsq{}Russia\PYZsq{}, \PYZsq{}Benjamen\PYZsq{}, \PYZsq{}Kenford\PYZsq{}, 51),
        (\PYZsq{}China\PYZsq{}, \PYZsq{}Etan\PYZsq{}, \PYZsq{}Burn\PYZsq{}, 53),
        (\PYZsq{}Russia\PYZsq{}, \PYZsq{}Merralee\PYZsq{}, \PYZsq{}Chaperlin\PYZsq{}, 38),
        (\PYZsq{}Indonesia\PYZsq{}, \PYZsq{}Lanny\PYZsq{}, \PYZsq{}Malam\PYZsq{}, 49),
        (\PYZsq{}Canada\PYZsq{}, \PYZsq{}Wilhelm\PYZsq{}, \PYZsq{}Deeprose\PYZsq{}, 54),
        (\PYZsq{}Czech Republic\PYZsq{}, \PYZsq{}Lari\PYZsq{}, \PYZsq{}Hillhouse\PYZsq{}, 48),
        (\PYZsq{}China\PYZsq{}, \PYZsq{}Ossie\PYZsq{}, \PYZsq{}Woodley\PYZsq{}, 52),
        (\PYZsq{}Macedonia\PYZsq{}, \PYZsq{}April\PYZsq{}, \PYZsq{}Tyer\PYZsq{}, 50),
        (\PYZsq{}Vietnam\PYZsq{}, \PYZsq{}Madelon\PYZsq{}, \PYZsq{}Dansey\PYZsq{}, 53),
        (\PYZsq{}Ukraine\PYZsq{}, \PYZsq{}Korella\PYZsq{}, \PYZsq{}McNamee\PYZsq{}, 52),
        (\PYZsq{}Jamaica\PYZsq{}, \PYZsq{}Linnea\PYZsq{}, \PYZsq{}Cannam\PYZsq{}, 43),
        (\PYZsq{}China\PYZsq{}, \PYZsq{}Mart\PYZsq{}, \PYZsq{}Coling\PYZsq{}, 52),
        (\PYZsq{}Indonesia\PYZsq{}, \PYZsq{}Marna\PYZsq{}, \PYZsq{}Causbey\PYZsq{}, 47),
        (\PYZsq{}China\PYZsq{}, \PYZsq{}Berni\PYZsq{}, \PYZsq{}Daintier\PYZsq{}, 55),
        (\PYZsq{}Poland\PYZsq{}, \PYZsq{}Cynthia\PYZsq{}, \PYZsq{}Hassell\PYZsq{}, 49),
        (\PYZsq{}Canada\PYZsq{}, \PYZsq{}Carma\PYZsq{}, \PYZsq{}Schule\PYZsq{}, 49),
        (\PYZsq{}Indonesia\PYZsq{}, \PYZsq{}Malia\PYZsq{}, \PYZsq{}Blight\PYZsq{}, 48),
        (\PYZsq{}China\PYZsq{}, \PYZsq{}Paulo\PYZsq{}, \PYZsq{}Seivertsen\PYZsq{}, 47),
        (\PYZsq{}Niger\PYZsq{}, \PYZsq{}Kaylee\PYZsq{}, \PYZsq{}Hearley\PYZsq{}, 54),
        (\PYZsq{}Japan\PYZsq{}, \PYZsq{}Maure\PYZsq{}, \PYZsq{}Jandak\PYZsq{}, 46),
        (\PYZsq{}Argentina\PYZsq{}, \PYZsq{}Foss\PYZsq{}, \PYZsq{}Feavers\PYZsq{}, 45),
        (\PYZsq{}Venezuela\PYZsq{}, \PYZsq{}Ron\PYZsq{}, \PYZsq{}Leggitt\PYZsq{}, 60),
        (\PYZsq{}Russia\PYZsq{}, \PYZsq{}Flint\PYZsq{}, \PYZsq{}Gokes\PYZsq{}, 40),
        (\PYZsq{}China\PYZsq{}, \PYZsq{}Linet\PYZsq{}, \PYZsq{}Conelly\PYZsq{}, 52),
        (\PYZsq{}Philippines\PYZsq{}, \PYZsq{}Nikolas\PYZsq{}, \PYZsq{}Birtwell\PYZsq{}, 57),
        (\PYZsq{}Australia\PYZsq{}, \PYZsq{}Eduard\PYZsq{}, \PYZsq{}Leipelt\PYZsq{}, 53)
\end{Verbatim}


    \paragraph{Using Python Variables in your SQL
Statements}\label{using-python-variables-in-your-sql-statements}

\subparagraph{You can use python variables in your SQL statements by
adding a ":" prefix to your python variable
names.}\label{you-can-use-python-variables-in-your-sql-statements-by-adding-a-prefix-to-your-python-variable-names.}

\subparagraph{\texorpdfstring{For example, if I have a python variable
\texttt{country} with a value of \texttt{"Canada"}, I can use this
variable in a SQL query to find all the rows of students from
Canada.}{For example, if I have a python variable country with a value of "Canada", I can use this variable in a SQL query to find all the rows of students from Canada.}}\label{for-example-if-i-have-a-python-variable-country-with-a-value-of-canada-i-can-use-this-variable-in-a-sql-query-to-find-all-the-rows-of-students-from-canada.}

    \begin{Verbatim}[commandchars=\\\{\}]
{\color{incolor}In [{\color{incolor} }]:} \PY{n}{country} \PY{o}{=} \PY{l+s+s2}{\PYZdq{}}\PY{l+s+s2}{Canada}\PY{l+s+s2}{\PYZdq{}}
        \PY{o}{\PYZpc{}}\PY{k}{sql} select * from INTERNATIONAL\PYZus{}STUDENT\PYZus{}TEST\PYZus{}SCORES where country = :country
\end{Verbatim}


    \paragraph{Assigning the Results of Queries to Python
Variables}\label{assigning-the-results-of-queries-to-python-variables}

    \subparagraph{You can use the normal python assignment syntax to assign
the results of your queries to python
variables.}\label{you-can-use-the-normal-python-assignment-syntax-to-assign-the-results-of-your-queries-to-python-variables.}

\subparagraph{\texorpdfstring{For example, I have a SQL query to
retrieve the distribution of test scores (i.e. how many students got
each score). I can assign the result of this query to the variable
\texttt{test\_score\_distribution} using the \texttt{=}
operator.}{For example, I have a SQL query to retrieve the distribution of test scores (i.e. how many students got each score). I can assign the result of this query to the variable test\_score\_distribution using the = operator.}}\label{for-example-i-have-a-sql-query-to-retrieve-the-distribution-of-test-scores-i.e.-how-many-students-got-each-score.-i-can-assign-the-result-of-this-query-to-the-variable-test_score_distribution-using-the-operator.}

    \begin{Verbatim}[commandchars=\\\{\}]
{\color{incolor}In [{\color{incolor} }]:} \PY{n}{test\PYZus{}score\PYZus{}distribution} \PY{o}{=} \PY{o}{\PYZpc{}}\PY{k}{sql} SELECT test\PYZus{}score as \PYZdq{}Test Score\PYZdq{}, count(*) as \PYZdq{}Frequency\PYZdq{} from INTERNATIONAL\PYZus{}STUDENT\PYZus{}TEST\PYZus{}SCORES GROUP BY test\PYZus{}score;
        \PY{n}{test\PYZus{}score\PYZus{}distribution}
\end{Verbatim}


    \paragraph{Converting Query Results to
DataFrames}\label{converting-query-results-to-dataframes}

    \subparagraph{\texorpdfstring{You can easily convert a SQL query result
to a pandas dataframe using the \texttt{DataFrame()} method. Dataframe
objects are much more versatile than SQL query result objects. For
example, we can easily graph our test score distribution after
converting to a
dataframe.}{You can easily convert a SQL query result to a pandas dataframe using the DataFrame() method. Dataframe objects are much more versatile than SQL query result objects. For example, we can easily graph our test score distribution after converting to a dataframe.}}\label{you-can-easily-convert-a-sql-query-result-to-a-pandas-dataframe-using-the-dataframe-method.-dataframe-objects-are-much-more-versatile-than-sql-query-result-objects.-for-example-we-can-easily-graph-our-test-score-distribution-after-converting-to-a-dataframe.}

    \begin{Verbatim}[commandchars=\\\{\}]
{\color{incolor}In [{\color{incolor} }]:} \PY{n}{dataframe} \PY{o}{=} \PY{n}{test\PYZus{}score\PYZus{}distribution}\PY{o}{.}\PY{n}{DataFrame}\PY{p}{(}\PY{p}{)}
        
        \PY{o}{\PYZpc{}}\PY{k}{matplotlib} inline
        \PY{k+kn}{import} \PY{n+nn}{seaborn}
        
        \PY{n}{plot} \PY{o}{=} \PY{n}{seaborn}\PY{o}{.}\PY{n}{barplot}\PY{p}{(}\PY{n}{x}\PY{o}{=}\PY{l+s+s1}{\PYZsq{}}\PY{l+s+s1}{Test Score}\PY{l+s+s1}{\PYZsq{}}\PY{p}{,}\PY{n}{y}\PY{o}{=}\PY{l+s+s1}{\PYZsq{}}\PY{l+s+s1}{Frequency}\PY{l+s+s1}{\PYZsq{}}\PY{p}{,} \PY{n}{data}\PY{o}{=}\PY{n}{dataframe}\PY{p}{)}
\end{Verbatim}


    Now you know how to work with Db2 from within JupyterLab notebooks using
SQL "magic"!

    \begin{Verbatim}[commandchars=\\\{\}]
{\color{incolor}In [{\color{incolor} }]:} \PY{o}{\PYZpc{}\PYZpc{}}\PY{k}{sql} 
        
        \PYZhy{}\PYZhy{} Feel free to experiment with the data set provided in this notebook for practice:
        SELECT country, first\PYZus{}name, last\PYZus{}name, test\PYZus{}score FROM INTERNATIONAL\PYZus{}STUDENT\PYZus{}TEST\PYZus{}SCORES;
\end{Verbatim}


    Copyright © 2018
\href{cognitiveclass.ai?utm_source=bducopyrightlink\&utm_medium=dswb\&utm_campaign=bdu}{cognitiveclass.ai}.
This notebook and its source code are released under the terms of the
\href{https://bigdatauniversity.com/mit-license/}{MIT License}.


    % Add a bibliography block to the postdoc
    
    
    
    \end{document}
